%%%%%%%%%%%%%%%%%%%%%%%%%%%%%%%%%%%%%%%%%%%%%%%%%%%%%%%%%%%%%%%%%%%%%%%%
%                                                                      %
%     File: Thesis_Abstract.tex                                        %
%     Tex Master: Thesis.tex                                           %
%                                                                      %
%     Author: Andre C. Marta                                           %
%     Last modified : 21 Jan 2011                                      %
%                                                                      %
%%%%%%%%%%%%%%%%%%%%%%%%%%%%%%%%%%%%%%%%%%%%%%%%%%%%%%%%%%%%%%%%%%%%%%%%

\section*{Abstract}

% Add entry in the table of contents as section
\addcontentsline{toc}{section}{Abstract}

%Insert your abstract here with a maximum of 250 words, followed by 4 to 6 keywords...
Advances in technology have introduced new Unmanned Aircraft (UA) into the market, that are able of operating in a wide variety of scenes and offer great capabilities, such as increased payload and time of operation. Adding to this, the fast growing availability of cheap small unmanned aircraft (sUA), which can offer similar features with added simplicity of operation, has brought an increased interest of the general public to aviation. This simplicity means that an operator is not required to possess any knowledge of the rules of the air, to be able to fly an sUA. The introduction of this kind of user into such a highly regulated world, as is aviation, highly increases the risk of air collision. Thus, the objective of this dissertation is to study and develop strategies and systems that allow the low cost incorporation of Sense and Avoid capabilities in small unmanned aircraft.  Accordingly, two solutions were explored: adaptation of a position and anti-collision lights system to an sUA, and development of an Infrared Sense and Avoid prototype which enables the detection of an intruder, providing its relative position and type of aircraft, so that possible avoidance maneuvers may be planned. A relevant contribution is the creation of a method to identify the type of aircraft using Morse code. Results show that the first prototype increases detectability while the second provides detection of aircraft with the same equipment.\\

\vfill

\textbf{\Large Keywords:} Infrared, See and Avoid, Sense and Avoid, sUA

\cleardoublepage

