%%%%%%%%%%%%%%%%%%%%%%%%%%%%%%%%%%%%%%%%%%%%%%%%%%%%%%%%%%%%%%%%%%%%%%%%
%                                                                      %
%     File: Thesis_Glossary.tex                                        %
%     Tex Master: Thesis.tex                                           %
%                                                                      %
%     Author: Andre C. Marta                                           %
%     Last modified : 30 Oct 2012                                      %
%                                                                      %
%%%%%%%%%%%%%%%%%%%%%%%%%%%%%%%%%%%%%%%%%%%%%%%%%%%%%%%%%%%%%%%%%%%%%%%%
%
% The definitions can be placed anywhere in the document body
% and their order is sorted by <symbol> automatically when
% calling makeindex in the makefile
%
% The \glossary command has the following syntax:
%
% \glossary{entry}
%
% The \nomenclature command has the following syntax:
%
% \nomenclature[<prefix>]{<symbol>}{<description>}
%
% where <prefix> is used for fine tuning the sort order,
% <symbol> is the symbol to be described, and <description> is
% the actual description.

% ----------------------------------------------------------------------

\glossary
{
	name={Aircraft},
	description={Any machine that can derive support in the atmosphere from the reactions of the air other than reactions of the air against the earth's surface \citep{TheEuropeanCommission2003}.}
}

\glossary
{
	name={Small Unmanned Aircraft},
	description={An unmanned aircraft weighing less than 25 kg, including everything that is on board the aircraft \citep{FederalAviationAdministration2015}.}
}

\glossary
{
	name={Unmanned Aircraft},
	description={An aircraft which is intended to operate with no pilot on board \citep{InternationalCivilAviationOrganization2011}.}
}

\glossary
{
	name={Segregated Airspace},
	description={Airspace of specified dimensions allocated for exclusive use to a specific user(s) \citep{InternationalCivilAviationOrganization2011}.}
}

\glossary
{
	name={Sense and Avoid},
	description={Term used to collectively describe the combined functions of separation provision and collision avoidance. It is the corresponding function of the analogous See and Avoid function performed by the flight-crew of manned aircraft \citep{Eurocontrol2010}.}
}

\glossary
{
	name={Separation Provision},
	description={The process of maintaining sufficient physical separation between aircraft, terrain and other objects so that there is no discernible risk of collision \citep{Eurocontrol2010}.}
}

\glossary
{
	name={Collision Avoidance},
	description={The process of avoiding a collision (with another aircraft, vehicle, structure or terrain). In general terms, collision avoidance is intended to operate when separation assurance cannot be maintained \citep{Eurocontrol2010}.}
}

\glossary
{
	name={See and Avoid},
	description={The process performed by flight-crew with the ability to visually detect other aircraft, terrain or objects in order to perform separation assurance or collision avoidance \citep{Eurocontrol2010}.}
}